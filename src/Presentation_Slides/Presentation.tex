\documentclass[11pt,aspectratio=16:9]{beamer}

% Theme and Color Scheme
\usetheme{Madrid}
\usecolortheme{default}
\setbeamertemplate{navigation symbols}{}
\setbeamertemplate{footline}[page number]

% Packages
\usepackage[utf8]{inputenc}
\usepackage[spanish]{babel}
\usepackage{graphicx}
\usepackage{booktabs}
\usepackage{amsmath,amssymb,amsfonts}
\usepackage{xcolor}
\usepackage{hyperref}
\usepackage{tikz}
\usetikzlibrary{shapes,arrows,positioning}

% Configure graphics path
\graphicspath{{images/}}

% Define custom colors
\definecolor{darkblue}{rgb}{0.1, 0.3, 0.6}
\definecolor{lightblue}{rgb}{0.8, 0.9, 1.0}
\definecolor{darkgreen}{rgb}{0.2, 0.5, 0.3}

% Title page configuration
\title{Architecture for Real-Time Air Quality Monitoring}
\subtitle{and Personalized Health Recommendations in Bogotá}
\author{Jose Alejandro Cortazar López \\ Johan Esteban Castaño Martínez \\ Stivel Pinilla Puerta}
\institute{Systems Engineering Program \\ Francisco José de Caldas District University}
\date{December 2025}

% Custom title page
\defbeamertemplate*{title page}{custom}[1][]{
  \vbox{}
  \vfill
  \begin{centering}
    \begin{beamercolorbox}[sep=8pt,center,#1]{title}
      \usebeamerfont{title}\inserttitle\par%
      \ifx\insertsubtitle\@empty%
      \else%
        \vskip0.25em%
        {\usebeamerfont{subtitle}\usebeamercolor[fg]{subtitle}\insertsubtitle\par}%
      \fi%
    \end{beamercolorbox}%
    \vskip1em\par
    \begin{beamercolorbox}[sep=8pt,center,#1]{author}
      \usebeamerfont{author}\insertauthor
    \end{beamercolorbox}
    \begin{beamercolorbox}[sep=8pt,center,#1]{institute}
      \usebeamerfont{institute}\insertinstitute
    \end{beamercolorbox}
    \begin{beamercolorbox}[sep=8pt,center,#1]{date}
      \usebeamerfont{date}\insertdate
    \end{beamercolorbox}
  \end{centering}
  \vfill
}

\setbeamertemplate{title page}[custom]

\begin{document}

% ===== SLIDE 1: TITLE PAGE =====
\begin{frame}
  \titlepage
\end{frame}

% ===== SLIDE 2: TABLE OF CONTENTS =====
\begin{frame}{Agenda}
  \tableofcontents
\end{frame}

% ===== SECTION 1: PROBLEM & CONTEXT =====
\section{Problem Statement}

\begin{frame}{Motivation: Air Quality Crisis in Bogotá}
  \begin{columns}[T]
    \column{0.5\textwidth}
    \begin{itemize}
      \item \textbf{Public Health Threat}: Air pollution affects millions globally
      \item \textbf{Fragmented Data}: Multiple disconnected monitoring sources
      \item \textbf{Limited Accessibility}: Citizens lack integrated real-time information
      \item \textbf{Absence of Personalization}: Generic alerts without health risk profiles
    \end{itemize}
    
    \column{0.5\textwidth}
    \begin{block}{Current Challenges}
      \begin{itemize}
        \item No unified air quality data platform
        \item Response latency from multiple APIs
        \item Difficult health decision-making
        \item Scalability concerns for city-wide deployment
      \end{itemize}
    \end{block}
  \end{columns}
\end{frame}

\begin{frame}{Problem Statement}
  \textbf{Core Problem:} Bogotá residents lack timely access to \alert{integrated} air quality data across multiple monitoring sources.
  
  \vspace{1em}
  
  \textbf{Specific Gaps:}
  \begin{enumerate}
    \item \textbf{Data Fragmentation}: AQICN, Google Maps API, IQAir operate independently
    \item \textbf{Performance Constraints}: High-latency queries unsuitable for real-time dashboards
    \item \textbf{Lack of Personalization}: No health recommendations tailored to individual profiles
    \item \textbf{Scalability Limitations}: Current systems cannot support city-wide deployment
  \end{enumerate}
  
  \vspace{1em}
  
  \textbf{Solution Requirement}: Production-ready architecture combining multi-source integration, normalized schema, sub-100ms queries, and personalized health guidance.
\end{frame}

% ===== SECTION 2: OBJECTIVES =====
\section{Objectives}

\begin{frame}{Primary \& Secondary Objectives}
  \textbf{Primary Objective:}
  \begin{block}{}
    Design and validate a \textit{production-ready} air quality monitoring architecture that integrates multi-source data, normalizes heterogeneous payloads into a unified 3NF schema, and delivers personalized health recommendations.
  \end{block}
  
  \vspace{1em}
  
  \textbf{Secondary Objectives:}
  \begin{enumerate}
    \item Data Integration: Aggregate AQICN, Google, IQAir with deduplication \& validation
    \item Database Design: Implement 3NF schema with temporal partitioning (\(<100\)ms queries)
    \item Personalization: Rule-based recommendations aligned with EPA/WHO standards
    \item Concurrency \& Scalability: Support 50--100+ concurrent users
    \item Replicability: Generalize for multi-city Latin American deployment
  \end{enumerate}
\end{frame}

% ===== SECTION 3: PROPOSED SOLUTION =====
\section{Proposed Solution}

\begin{frame}{System Architecture Overview}
  \begin{center}
    \includegraphics[width=\textwidth]{fig1_architecture}
  \end{center}
  
  \vspace{0.5em}
  
  \textit{4-layer architecture: Ingestion → Persistence → Application → Presentation}
\end{frame}

\begin{frame}{Layer 1: Ingestion}
  \textbf{Python APScheduler-based Data Pipeline}
  
  \vspace{0.5em}
  
  \begin{itemize}
    \item \textbf{Polling Frequency}: 10-minute intervals
    \item \textbf{Data Sources}: 
      \begin{itemize}
        \item AQICN (Air Quality Index)
        \item Google Maps Air Quality API
        \item IQAir (Premium air quality data)
      \end{itemize}
    \item \textbf{Output}: 216 readings/hour (36 readings × 6 cycles)
    \item \textbf{Coverage}: 6 monitoring stations × 6 pollutants (PM$_{2.5}$, PM$_{10}$, NO$_2$, O$_3$, SO$_2$, CO)
    \item \textbf{Processing}: JSON validation via Pydantic, deduplication, idempotent upserts
  \end{itemize}
  
  \begin{block}{Key Advantage}
    Zero MVCC contention enables continuous ingestion without blocking analytical queries
  \end{block}
\end{frame}

\begin{frame}{Layer 2: Persistence (Database)}
  \textbf{PostgreSQL 12+ with 3NF Normalized Schema}
  
  \vspace{0.5em}
  
  \begin{columns}[T]
    \column{0.5\textwidth}
    \textbf{8 Core Entities:}
    \begin{itemize}
      \item Station
      \item Pollutant
      \item Provider
      \item AirQualityReading
      \item AppUser
      \item Alert
      \item Recommendation
      \item ProductRecommendation
    \end{itemize}
    
    \column{0.5\textwidth}
    \textbf{Optimization Techniques:}
    \begin{itemize}
      \item Temporal partitioning (monthly chunks)
      \item Composite B-tree indexes
      \item Materialized views for aggregates
      \item Constraint exclusion pruning
      \item MongoDB for schema-flexible user preferences
    \end{itemize}
  \end{columns}
\end{frame}

\begin{frame}{Layer 3: Application}
  \textbf{FastAPI REST API with Intelligent Caching}
  
  \vspace{0.5em}
  
  \begin{itemize}
    \item \textbf{Framework}: FastAPI (Python async framework)
    \item \textbf{Caching}: Redis with 5--10 minute TTL
    \item \textbf{Connection Pooling}: PgBouncer (50 connections)
    \item \textbf{Recommendation Engine}: 
      \begin{itemize}
        \item Rule-based health guidance
        \item Personalization by age, respiratory conditions, activity level
        \item Mapped to EPA/WHO AQI standards
      \end{itemize}
    \item \textbf{Endpoints}: 
      \begin{itemize}
        \item Latest readings by station
        \item Historical trends (7-day, 30-day, 1-year)
        \item User-specific alerts and recommendations
      \end{itemize}
  \end{itemize}
\end{frame}

\begin{frame}{Layer 4: Presentation}
  \textbf{Interactive User Dashboard}
  
  \vspace{0.5em}
  
  \textbf{Technology Stack}: React/Vue.js with responsive design
  
  \vspace{0.5em}
  
  \textbf{Key Features:}
  \begin{itemize}
    \item Real-time AQI display with color-coded severity (Green → Red)
    \item Interactive time-series charts (7-day, 30-day, 1-year trends)
    \item Geospatial heatmaps showing pollution hotspots
    \item Personalized health recommendations
    \item User alerts and notification center
    \item Mobile-responsive design (Android/iOS compatible)
  \end{itemize}
\end{frame}

% ===== SECTION 4: TECHNICAL IMPLEMENTATION =====
\section{Technical Details}

\begin{frame}{Database Schema: Normalization (3NF)}
  \begin{center}
    \begin{tabular}{lp{5cm}}
      \toprule
      \textbf{Entity} & \textbf{Description} \\
      \midrule
      Station & Geographic location with lat/lon, city metadata \\
      Pollutant & PM2.5, PM10, NO2, O3, SO2, CO with WHO limits \\
      Provider & AQICN, Google, IQAir with API credentials \\
      AirQualityReading & Timestamped measurements (Station, Pollutant, Provider, Value) \\
      AppUser & Citizen profiles with health profile (age, respiratory conditions) \\
      Alert & User-configured thresholds (e.g., PM2.5 > 75 microg/m3) \\
      Recommendation & Health guidance rules mapped to AQI levels \\
      ProductRec. & Protective products (masks, air purifiers) per AQI level \\
      \bottomrule
    \end{tabular}
  \end{center}
  
  \begin{block}{Design Principle}
    Eliminates redundancy while maintaining referential integrity through composite primary/foreign keys
  \end{block}
\end{frame}

\begin{frame}{Query Optimization Strategies}
  \textbf{Performance Targets}: Sub-100ms latency on 85,000+ readings
  
  \vspace{1em}
  
  \begin{columns}[T]
    \column{0.5\textwidth}
    \textbf{Indexing Strategy:}
    \begin{itemize}
      \item Composite B-tree: (Station, Timestamp, Pollutant)
      \item Partial indexes on active alerts
      \item BRIN indexes on temporal columns
      \item Hash indexes on provider IDs
    \end{itemize}
    
    \column{0.5\textwidth}
    \textbf{Partitioning \& Views:}
    \begin{itemize}
      \item Range partitioning by month
      \item Constraint exclusion pruning
      \item Materialized views for daily aggregates
      \item 35× row reduction (85K → 2.4K)
    \end{itemize}
  \end{columns}
\end{frame}

\begin{frame}{Query Optimization: Example}
  \textbf{Query 1: Latest readings by station}
  
  \vspace{1em}
  
  \begin{block}{SQL Query}
    \small
    \texttt{SELECT s.name, p.name, aqr.value, aqr.timestamp \\
    FROM AirQualityReading aqr \\
    JOIN Station s ON aqr.station\_id = s.id \\
    JOIN Pollutant p ON aqr.pollutant\_id = p.id \\
    WHERE aqr.timestamp $>$ NOW() - INTERVAL '1 hour'\\
    ORDER BY aqr.timestamp DESC LIMIT 1;}
  \end{block}
  
  \vspace{1em}
  
  \textbf{Performance}: \textcolor{darkgreen}{42.8 ms} (target: $<$50 ms) [OK]
  \begin{itemize}
    \item Composite index enables single-pass scan
    \item Constraint exclusion limits partitions scanned
  \end{itemize}
\end{frame}

% ===== SECTION 5: RESULTS =====
\section{Results \& Validation}

\begin{frame}{Query Performance Validation}
  \textbf{Baseline}: 85,000 air quality readings across 6 stations and 6 pollutants
  
  \vspace{1em}
  
  \begin{center}
    \begin{tabular}{llll}
      \toprule
      \textbf{Query ID} & \textbf{Description} & \textbf{Latency} & \textbf{Target} \\
      \midrule
      Q1 & Latest readings/station & 42.8 ms & <50 ms [OK] \\
      Q2 & Monthly historical avg & 127.3 ms & <150 ms [OK] \\
      Q3 & Active alerts (7-day) & 143.6 ms & <150 ms [OK] \\
      Q4 & 24-hour completeness & 87.5 ms & <100 ms [OK] \\
      Q5 & User recommendations & 73.9 ms & <80 ms [OK] \\
      \bottomrule
    \end{tabular}
  \end{center}
  
  \vspace{1em}
  
  \textit{All 5 core queries meet or exceed performance targets}
\end{frame}

\begin{frame}{Optimization Impact: Before vs. After}
  \textbf{Effect of Temporal Partitioning:}
  
  \vspace{1em}
  
  \begin{columns}[T]
    \column{0.5\textwidth}
    \textbf{Point Query (Q1):}
    \begin{itemize}
      \item Without partitioning: 48.2 ms
      \item With partitioning: 42.8 ms
      \item \textcolor{darkgreen}{Improvement: 11.2\%}
    \end{itemize}
    
    \column{0.5\textwidth}
    \textbf{Range Query (Q2):}
    \begin{itemize}
      \item Without partitioning: 182 ms
      \item With partitioning: 127.3 ms
      \item \textcolor{darkgreen}{Improvement: 30.2\%}
    \end{itemize}
  \end{columns}
  
  \vspace{1.5em}
  
  \textbf{Materialized Views Impact:}
  \begin{itemize}
    \item Raw readings: 85,000 rows
    \item Aggregated (daily): 2,400 rows
    \item Improvement: \textcolor{darkgreen}{35× reduction}
  \end{itemize}
\end{frame}

\begin{frame}{Concurrency Validation: 4 Scenarios}
  \textbf{Scenario 1: Ingestion vs. Dashboard Reads}
  \begin{itemize}
    \item MVCC isolation prevents read blocking
    \item Dashboard latency: <100 ms maintained [OK]
  \end{itemize}
  
  \vspace{0.5em}
  
  \textbf{Scenario 2: Multiple Concurrent Dashboards}
  \begin{itemize}
    \item 100 concurrent users → 140 req/sec throughput
    \item Read-heavy, low-contention scenario [OK]
  \end{itemize}
  
  \vspace{0.5em}
  
  \textbf{Scenario 3: Batch Aggregation Jobs}
  \begin{itemize}
    \item Daily 1 AM off-peak aggregation
    \item Completes in <30 seconds [OK]
  \end{itemize}
  
  \vspace{0.5em}
  
  \textbf{Scenario 4: Hot Data Queries}
  \begin{itemize}
    \item Recent timestamp filters at 500 concurrent users
    \item Partial indexes maintain <150 ms latency [OK]
  \end{itemize}
\end{frame}

\begin{frame}{Scalability Projections}
  \textbf{Current Deployment (Baseline):}
  \begin{itemize}
    \item Peak concurrent users: 50--100
    \item CPU utilization: 70--75\%
    \item Hardware: 4 vCPU, 16 GB RAM
  \end{itemize}
  
  \vspace{1em}
  
  \textbf{Scaling Predictions:}
  
  \begin{center}
    \begin{tabular}{lll}
      \toprule
      \textbf{Scale} & \textbf{Users} & \textbf{Configuration} \\
      \midrule
      Current & 50--100 & 4 vCPU, 16 GB \\
      Vertical & 500--1000 & 8+ vCPU, 32+ GB \\
      Distributed & 10K+ & Read replicas + sharding \\
      \bottomrule
    \end{tabular}
  \end{center}
  
  \vspace{1em}
  
  \textbf{Data Growth:} 3.06M rows (3 years) → 10 GB storage (10 years) with partitioning
\end{frame}

% ===== SECTION 6: CONCLUSIONS =====
\section{Conclusions}

\begin{frame}{Key Achievements}
  \begin{enumerate}
    \item \textbf{Normalized 3NF Schema}: 8-entity design enforcing referential integrity
    
    \item \textbf{Query-Optimized Performance}: Sub-100ms latencies validated through PostgreSQL EXPLAIN ANALYZE
    
    \item \textbf{Multi-Source Integration}: 216 readings/hour sustained without MVCC contention
    
    \item \textbf{Transparent Personalization}: Rule-based health guidance mapped to EPA/WHO standards
    
    \item \textbf{Production-Ready Concurrency}: 4 realistic scenarios validated; supports 500 concurrent users at <150ms latency
  \end{enumerate}
\end{frame}

\begin{frame}{Future Work}
  \textbf{Short-term (6 months):}
  \begin{itemize}
    \item TimescaleDB hypertables for native time-series optimization
    \item MinIO object storage for raw JSON archival and audit trails
  \end{itemize}
  
  \vspace{0.5em}
  
  \textbf{Medium-term (12 months):}
  \begin{itemize}
    \item PostGIS spatial interpolation and heatmap generation
    \item Machine learning forecasting (LSTM/GRU for 24-hour pollution predictions)
  \end{itemize}
  
  \vspace{0.5em}
  
  \textbf{Long-term (18+ months):}
  \begin{itemize}
    \item High availability with read replicas and automated failover (99.9\% uptime)
    \item Multi-city scaling to Medellín, Cali, and other Latin American hubs
  \end{itemize}
\end{frame}

\begin{frame}{Broader Impact}
  \begin{block}{Scientific Contribution}
    Provides a \textit{replicable, open-source foundation} for urban air quality monitoring in resource-constrained regions
  \end{block}
  
  \begin{block}{Societal Impact}
    \begin{itemize}
      \item Empowers citizens with transparent, explainable health recommendations
      \item Eliminates proprietary black-box algorithms
      \item Supports evidence-based urban planning and pollution mitigation policies
    \end{itemize}
  \end{block}
  
  \begin{block}{Scalability}
    Generalizable to other Latin American cities facing similar air quality challenges
  \end{block}
\end{frame}

\begin{frame}[plain,c]
  \begin{center}
    {\Huge Thank You}
    
    \vspace{2em}
    
    Questions? \\
    \vspace{1em}
    
    {\small \texttt{Contact: project@example.com}}
  \end{center}
\end{frame}

\end{document}
