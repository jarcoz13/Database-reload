\chapter{Background}
\label{ch:background}

This chapter establishes the business and technical context for the air quality monitoring platform. It covers the importance of air quality monitoring as a public health challenge in urban environments, summarizes the key functional and non-functional requirements that guided system design, and reviews existing data platforms and related work to position this project's contributions within the broader landscape of environmental health informatics.

%%%%%%%%%%%%%%%%%%%%%%%%%%%%%%%%%%%%%%%%%%%%%%%%%%%%%%%%%%%%%%%%%%%%%%%%%%%%%%%%%%%
\section{Business Context: Air Quality Monitoring in Bogotá}
\label{sec:bg_business}

Air pollution is one of the most significant environmental risk factors for global mortality. The World Health Organization estimates that ambient and household air pollution jointly cause 7--8 million premature deaths annually, with 99\% of the world's population exposed to air exceeding WHO guideline values~\citep{whopollution}. The 2024 State of Global Air report identifies fine particulate matter (PM$_{2.5}$) as the second leading risk factor for mortality worldwide, surpassing tobacco smoking and high blood pressure~\citep{state}.

\subsection{Air Quality Challenge in Bogotá}
\label{subsec:bg_bogota}

Bogotá, Colombia's capital and home to over 8 million residents, faces particularly acute air quality challenges. Latin American megacities struggle with rapid urbanization, heavy vehicular emissions, industrial activities, and geographic conditions that trap pollutants. In Bogotá, long-term studies document persistent spatially heterogeneous PM$_{2.5}$ concentrations, particularly in industrial zones and high-traffic corridors.

While the city has achieved incremental improvements—declining from 15.7~$\mu$g/m$^3$ in 2017 to 13.1~$\mu$g/m$^3$ in 2019 through the Air Plan 2030 initiative—current levels still exceed WHO recommended annual exposure limits of 5~$\mu$g/m$^3$. Vulnerable populations—children, elderly individuals, and people with respiratory or cardiovascular conditions—face disproportionate health risks from chronic exposure.

\subsection{Market Opportunity and Stakeholder Needs}
\label{subsec:bg_opportunity}

Multiple air quality data sources exist (AQICN, Google Air Quality API, IQAir), yet citizens and policymakers lack integrated, timely, and personalized access to this information. Existing platforms present barriers:

\begin{itemize}
    \item \textbf{Fragmentation:} Data platforms operate independently with inconsistent formats, units, and update frequencies, requiring users to navigate multiple interfaces.
    
    \item \textbf{Lack of Personalization:} Platforms provide aggregate indices without translating air quality data into health guidance tailored to individual conditions, age groups, or planned activities.
    
    \item \textbf{Technical Barriers:} Citizens without environmental science expertise struggle to interpret pollutant abbreviations and AQI scales, particularly affecting vulnerable populations most needing clear guidance.
    
    \item \textbf{Operational Constraints:} Proprietary platforms enforce strict API quotas and tiered pricing, complicating continuous city-scale monitoring by researchers and municipalities.
    
    \item \textbf{Real-Time Gaps:} Many platforms aggregate data hourly or longer, missing rapid pollution events warranting immediate health warnings.
\end{itemize}

This project addresses these gaps by designing a unified platform integrating heterogeneous data sources, normalizing measurements into a consistent schema, and delivering personalized, evidence-based health recommendations to citizens. The platform serves multiple stakeholder groups: citizens seeking health guidance, researchers conducting longitudinal analysis, policy makers monitoring urban conditions, and technical administrators managing data operations.

%%%%%%%%%%%%%%%%%%%%%%%%%%%%%%%%%%%%%%%%%%%%%%%%%%%%%%%%%%%%%%%%%%%%%%%%%%%%%%%%%%%
\section{Functional and Non-Functional Requirements}
\label{sec:bg_requirements}

To address the identified gaps, this project defines a comprehensive set of functional and non-functional requirements refined iteratively through three project workshops, aligned with the Delivery 3 baseline scope. For complete details on all 14 functional requirements and 17 non-functional requirements, see Appendix~\ref{ch:app_requirements}.

\subsection{Core Functional Requirements}

The ten critical baseline functional requirements are:

\begin{itemize}
    \item \textbf{FR1 -- Periodic Data Collection:} Automated ingestion from AQICN, Google Air Quality API, and IQAir at 10--60 minute intervals, with normalization, validation, and failure logging.
    
    \item \textbf{FR2 -- Historical Data Access:} Query and visualization of historical data (minimum 3 years) filtered by date range, location, and pollutant, with export to CSV/JSON.
    
    \item \textbf{FR3 -- Unified Data Presentation:} Display air quality information in consistent format independent of source API.
    
    \item \textbf{FR4 -- KPI Dashboards:} Real-time dashboards displaying current AQI, main pollutant, and trend charts updated at ingestion intervals.
    
    \item \textbf{FR5 -- Custom Report Generation:} User-configurable reports with date, location, and pollutant filters, downloadable in standard formats.
    
    \item \textbf{FR6 -- Interactive Visualization:} Time-series graphs with user controls for date range and pollutant selection.
    
    \item \textbf{FR8 -- Rule-Based Recommendations:} Deterministic health guidance based on AQI thresholds aligned with EPA and WHO guidelines.
    
    \item \textbf{FR9 -- Configurable Alerts:} User-defined alerts triggering when AQI exceeds thresholds, delivered via email or in-app.
    
    \item \textbf{FR12 -- Geographic Search:} Search air quality data by country, city, or region.
    
    \item \textbf{FR13 -- Responsive Web Interface:} Mobile, tablet, and desktop compatibility with WCAG 2.1 Level AA accessibility.
\end{itemize}

\subsection{Core Non-Functional Requirements}

The ten critical baseline non-functional targets are:

\begin{itemize}
    \item \textbf{NFR1/NFR2 -- Query Latency:} Dashboard queries < 2 seconds; historical queries over months < 5 seconds.
    
    \item \textbf{NFR3 -- Ingestion Throughput:} Process and persist $\geq$ 1,000 readings per 10-minute cycle.
    
    \item \textbf{NFR5 -- System Uptime:} Target 99.5\% uptime with graceful degradation if external APIs fail.
    
    \item \textbf{NFR6 -- Data Integrity:} Maintain referential integrity; detect and deduplicate duplicate readings.
    
    \item \textbf{NFR8 -- Audit Trail:} Log all ingestion activities with source, timestamp, and result status.
    
    \item \textbf{NFR9 -- Concurrent Users:} Support up to 1,000 concurrent web users without latency degradation.
    
    \item \textbf{NFR10 -- Data Volume:} Design database to scale to 10+ million readings across years and stations.
    
    \item \textbf{NFR12 -- Responsive Design:} Interface responsive on all viewport sizes; WCAG 2.1 Level AA compliant.
    
    \item \textbf{NFR16 -- Data Retention:} Retain historical air quality data 3+ years for research and longitudinal analysis.
\end{itemize}

Appendix~\ref{ch:app_requirements} provides the complete traceability matrix mapping requirements to 14 user stories and system components.

%%%%%%%%%%%%%%%%%%%%%%%%%%%%%%%%%%%%%%%%%%%%%%%%%%%%%%%%%%%%%%%%%%%%%%%%%%%%%%%%%%%
\section{Related Work: Existing Air Quality Platforms}
\label{sec:bg_related_work}

Multiple platforms provide air quality data to the public, each with distinct capabilities and limitations that informed this project's design decisions.

\subsection{AQICN (Air Quality Index China Network)}
\label{subsec:bg_aqicn}

AQICN offers one of the most comprehensive global databases, providing minute-level AQI readings and historical archives for over 100 countries~\citep{aqicn}. The platform aggregates data from government stations, low-cost sensor networks, and satellite observations, with historical CSV data available since 2015.

However, AQICN presents raw values without health context or personalization. Users must independently interpret AQI and implications for their conditions. The platform's strength lies in comprehensive spatial-temporal coverage rather than user-oriented decision support.

\subsection{Google Air Quality API}
\label{subsec:bg_google}

Google's Air Quality API provides high-resolution (500-meter grid) indices with pollutant concentrations and basic health recommendations~\citep{google}. The API returns structured JSON with current conditions, hourly forecasts, and activity-specific health tips.

While offering superior spatial granularity and accessible health messaging, Google's service enforces strict quota limits complicating city-scale analytics. Free-tier usage allows limited daily requests; high-volume access requires enterprise agreements. The service focuses on current and short-term forecasts, with limited long-term historical analysis support.

\subsection{IQAir AirVisual}
\label{subsec:bg_iqair}

IQAir operates a calibrated sensor network providing data through both consumer and REST API~\citep{iqairapi}. The platform emphasizes sensor accuracy and calibration, offering superior reliability to many low-cost networks, with hourly updates for major metropolitan areas.

IQAir's tiered pricing model presents barriers for research projects and non-commercial applications requiring comprehensive historical data. Hourly aggregation may miss rapid pollution events requiring finer temporal granularity.

\subsection{Academic and Related Research}

Environmental health informatics research explores machine learning for pollution forecasting and health impact prediction~\citep{esr}. While promising, operational deployment requires reliable data infrastructure and citizen-facing applications—the focus of this project. Low-cost sensor networks expand coverage beyond government stations but introduce data quality challenges requiring calibration protocols~\citep{wiley}. Smart city initiatives increasingly incorporate air quality monitoring, though most focus on municipal planning visualization rather than personalized citizen guidance~\citep{mdpi}.

\subsection{Gap Analysis and Project Contribution}

Existing platforms successfully aggregate air quality data but lack integration, personalization, and explainable recommendations. This project bridges the gap by:

\begin{enumerate}
    \item Integrating heterogeneous sources into a unified relational schema with normalization and deduplication.
    \item Providing personalized, rule-based recommendations aligned with EPA and WHO guidelines, ensuring transparency.
    \item Delivering a scalable architecture suitable for city-scale deployment and multi-city expansion.
    \item Enabling fast analytical queries (< 2--5 seconds) through indexed, partitioned schemas and materialized views.
    \item Creating clear separation of concerns (data, API, presentation) enabling independent scaling.
\end{enumerate}

The design balances operational simplicity with capability, avoiding unnecessary complexity while maintaining clear upgrade paths to advanced features (stream processing, machine learning, multi-region deployment) documented as future work.

%%%%%%%%%%%%%%%%%%%%%%%%%%%%%%%%%%%%%%%%%%%%%%%%%%%%%%%%%%%%%%%%%%%%%%%%%%%%%%%%%%%
\section{Summary of Chapter}
\label{sec:bg_summary}

This chapter established the business case for integrating air quality data from multiple sources into a unified platform delivering personalized recommendations to Bogotá's citizens. The key functional and non-functional requirements were summarized, with complete details in Appendix~\ref{ch:app_requirements}. A review of existing platforms and related research highlighted gaps that this project addresses: lack of integration, personalization, and explainable recommendations. The following chapters describe the system and database architecture (Chapter~3) and the design methodology (Chapter~4) that implement these requirements.
