%Two resources useful for abstract writing.
% Guidance of how to write an abstract/summary provided by Nature: https://cbs.umn.edu/sites/cbs.umn.edu/files/public/downloads/Annotated_Nature_abstract.pdf %https://writingcenter.gmu.edu/guides/writing-an-abstract
\chapter*{\center \Large  Abstract}
%%%%%%%%%%%%%%%%%%%%%%%%%%%%%%%%%%%%%%
% Abstract: 200-250 words summarizing problem, approach, design decisions, and key results
%%%%%%%%%%%%%%%%%%%%%%%%%%%%%%%%%%%

Air pollution continues to be a major public health challenge, particularly in large Latin American cities such as Bogotá, where PM$_{2.5}$ concentrations frequently exceed WHO exposure guidelines. Citizens and policymakers lack access to integrated, timely, and personalized air quality information despite the existence of multiple data sources (AQICN, Google Air Quality API, IQAir). This fragmentation creates barriers to informed decision-making about outdoor activities and health precautions, particularly affecting vulnerable populations including children, elderly individuals, and those with respiratory conditions.

This report presents a practical and reproducible architecture for addressing this gap by integrating periodic air quality data from multiple providers, normalizing records into a unified relational schema, and delivering personalized, rule-based health recommendations to citizens in Bogotá. The baseline implementation centers on PostgreSQL with declarative temporal partitioning and materialized views, combined with a lightweight NoSQL store for user preferences and dashboard configuration. A Python-based periodic ingestion pipeline (10-minute cycles) normalizes heterogeneous API payloads and performs batched inserts aligned with temporal partitions. The API layer exposes REST endpoints for dashboards and recommendations; the recommendation logic maps AQI thresholds and basic user metadata to evidence-based health guidance aligned with EPA and WHO guidelines.

The design achieves performance targets suitable for city-scale deployment: sub-2-second dashboard query response times over datasets exceeding one million records, support for up to 1,000 concurrent users, and 10-minute data freshness. Key contributions include a documented normalized schema (Third Normal Form) with optimized indexing and partitioning strategies, a unified data ingestion and normalization pipeline, and a transparent, explainable recommendation system. Advanced components—including dedicated object storage for raw payloads (MinIO), TimescaleDB time-series extensions, and machine learning models—are documented as future enhancements. This work establishes a foundation for scaling to multi-city deployments and integrating advanced analytics capabilities.

~\\[0.5cm]
\noindent % Provide your key words
\textbf{Keywords:} Air Quality Monitoring, PostgreSQL Partitioning, Materialized Views, Data Normalization, Rule-based Recommendations, Environmental Health Informatics

\vfill
\noindent
\textbf{Report's total word count:} Approximately 10,000-15,000 words (starting from Chapter 1 and finishing at the end of the conclusions chapter, excluding references, appendices, abstract, text in figures, tables, listings, and captions). \newline
\newline
\noindent
\textbf{Source Code Repository:} \url{https://github.com/DarcanoS/Database-II} \newline
\newline
This report was submitted as part of the Databases II course requirements at Francisco José de Caldas District University, Faculty of Engineering, Systems Engineering Program.

